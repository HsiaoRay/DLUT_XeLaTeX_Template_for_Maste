% !TEX TS-program = XeLaTeX
% !TEX encoding = UTF-8 Unicode

%%%%%%%%%%%%%%%%%%%%%%%%%%%%%%%%%%%%%%%%%%%%%%%%%%%%%%%%%%%%%%%%%%%%%% 
% 
%	大连理工大学硕士论文 XeLaTeX 模版 —— 格式文件 format.tex
%	版本:0.71
%	最后更新:2010.12.22
%	修改者:Yuri (E-mail: yuri_1985@163.com)
%	编译环境:Ubuntu 10.04 + TeXLive 2010 + TeXworks
% Windows XP SP3 + CTeXLive 2009 + WinEdt 5.6
% 
%%%%%%%%%%%%%%%%%%%%%%%%%%%%%%%%%%%%%%%%%%%%%%%%%%%%%%%%%%%%%%%%%%%%%% 

%%%%%%%%%%%%%%%%%%%%%%%%%%%%%%%%%%%%%%%%%%%%%%%%%%%%%%%%%%%%%%%%%%%%%% 
% 页面设置
%%%%%%%%%%%%%%%%%%%%%%%%%%%%%%%%%%%%%%%%%%%%%%%%%%%%%%%%%%%%%%%%%%%%%% 
% A4 纸张
\setlength{\paperwidth}{21.0cm}
\setlength{\paperheight}{29.7cm}
% 设置正文尺寸大小
\setlength{\textwidth}{16.1cm}
\setlength{\textheight}{22.2cm}
% 设置正文区在正中间
\newlength \mymargin
\setlength{\mymargin}{(\paperwidth-\textwidth)/2}
\setlength{\oddsidemargin}{(\mymargin)-1in}
\setlength{\evensidemargin}{(\mymargin)-1in}
% 设置正文区偏移量,奇数页向右偏,偶数页向左偏
\newlength \myshift
\setlength{\myshift}{0.35cm}	% 双面打印的奇偶页偏移值,可根据需要修改,建议小于 0.5cm
\addtolength{\oddsidemargin}{\myshift}
\addtolength{\evensidemargin}{-\myshift}
% 页眉页脚相关距离设置
\setlength{\topmargin}{-0.05cm}
\setlength{\headheight}{0.50cm}
\setlength{\headsep}{0.90cm}
\setlength{\footskip}{1.47cm}
% 公式的精调
\allowdisplaybreaks[4]  % 可以让公式在排不下的时候分页排,这可避免页面有大段空白。

% 下面这组命令使浮动对象的缺省值稍微宽松一点,从而防止幅度
% 对象占据过多的文本页面,也可以防止在很大空白的浮动页上放置很小的图形。
\renewcommand{\topfraction}{0.9999999}
\renewcommand{\textfraction}{0.0000001}
\renewcommand{\floatpagefraction}{0.9999}

%%%%%%%%%%%%%%%%%%%%%%%%%%%%%%%%%%%%%%%%%%%%%%%%%%%%%%%%%%%%%%%%%%%%%% 
% 字体字号定义
%%%%%%%%%%%%%%%%%%%%%%%%%%%%%%%%%%%%%%%%%%%%%%%%%%%%%%%%%%%%%%%%%%%%%% 
% 字号
\newcommand{\yihao}{\fontsize{26pt}{39pt}\selectfont}	  % 一号,1.5  倍行距
\newcommand{\xiaoyi}{\fontsize{24pt}{30pt}\selectfont}  % 小一,1.25 倍行距
\newcommand{\erhao}{\fontsize{22pt}{27.5pt}\selectfont} % 二号,1.25 倍行距
\newcommand{\xiaoer}{\fontsize{18pt}{22.5pt}\selectfont}% 小二,1.25 倍行距
\newcommand{\sanhao}{\fontsize{16pt}{20pt}\selectfont}  % 三号,1.25 倍行距
\newcommand{\xiaosan}{\fontsize{15pt}{19pt}\selectfont} % 小三,1.25 倍行距
\newcommand{\sihao}{\fontsize{14pt}{17.5pt}\selectfont} % 四号,1.25倍行距
\newcommand{\xiaosi}{\fontsize{12pt}{15pt}\selectfont}  % 小四,1.25倍行距
\newcommand{\dawu}{\fontsize{10.5pt}{18pt}\selectfont}  % 五号,1.75倍行距
\newcommand{\zhongwu}{\fontsize{10.5pt}{16pt}\selectfont}% 五号,1.5 倍行距
\newcommand{\wuhao}{\fontsize{10.5pt}{10.5pt}\selectfont}% 五号,单倍行距
\newcommand{\xiaowu}{\fontsize{9pt}{9pt}\selectfont}	   % 小五,单倍行距

\newcommand{\song}{\CJKfamily{song}}
\newcommand{\hei}{\CJKfamily{hei}}
\newcommand{\kai}{\CJKfamily{kai}}
\newcommand{\fs}{\CJKfamily{fs}}
\newcommand{\xkai}{\CJKfamily{xkai}}

% defaultfont 默认字体命令
\def\defaultfont{\renewcommand{\baselinestretch}{1.27}
  \fontsize{12pt}{15pt}\selectfont}

% 设置目录字体和行间距
\def\defaultmenufont{\renewcommand{\baselinestretch}{1.22}
  \fontsize{12pt}{15pt}\selectfont}

% 固定距离内容填入及下划线
\makeatletter
\newcommand\fixeddistanceleft[2][1cm]{{\hb@xt@ #1{#2\hss}}}
\newcommand\fixeddistancecenter[2][1cm]{{\hb@xt@ #1{\hss#2\hss}}}
\newcommand\fixeddistanceright[2][1cm]{{\hb@xt@ #1{\hss#2}}}
\newcommand\fixedunderlineleft[2][1cm]{\underline{\hb@xt@ #1{#2\hss}}}
\newcommand\fixedunderlinecenter[2][1cm]{\underline{\hb@xt@ #1{\hss#2\hss}}}
\newcommand\fixedunderlineright[2][1cm]{\underline{\hb@xt@ #1{\hss#2}}}
\makeatother

%%%%%%%%%%%%%%%%%%%%%%%%%%%%%%%%%%%%%%%%%%%%%%%%%%%%%%%%%%%%%%%%%%%%%% 
% 标题环境相关
%%%%%%%%%%%%%%%%%%%%%%%%%%%%%%%%%%%%%%%%%%%%%%%%%%%%%%%%%%%%%%%%%%%%%% 
% 定义、定理等环境
\theoremstyle{plain}
\theoremheaderfont{\hei\bf}
\theorembodyfont{\song\rmfamily}
\newtheorem{definition}{\hei 定义}[chapter]
\newtheorem{example}{\hei 例}[chapter]
\newtheorem{algorithm}{\hei 算法}[chapter]
\newtheorem{theorem}{\hei 定理}[chapter]
\newtheorem{axiom}{\hei 公理}[chapter]
\newtheorem{proposition}[theorem]{\hei 命题}
\newtheorem{property}{\hei 性质}
\newtheorem{lemma}[theorem]{\hei 引理}
\newtheorem{corollary}{\hei 推论}[chapter]
\newtheorem{remark}{\hei 注解}[chapter]
\newenvironment{proof}{\hei{证明} }{\hfill $\square$ \vskip 4mm}

% 目录标题
\renewcommand\contentsname{\hfill 目  录 \hfill}
\renewcommand\listfigurename{\hfill 插~图~目~录 \hfill}
\renewcommand\listtablename{\hfill 表~格~目~录 \hfill}
\renewcommand{\bibname}{\hfill 参~考~文~献 \hfill}

%%%%%%%%%%%%%%%%%%%%%%%%%%%%%%%%%%%%%%%%%%%%%%%%%%%%%%%%%%%%%%%%%%%%%% 
% 段落章节相关
%%%%%%%%%%%%%%%%%%%%%%%%%%%%%%%%%%%%%%%%%%%%%%%%%%%%%%%%%%%%%%%%%%%%%% 
\setcounter{secnumdepth}{4}
\setcounter{tocdepth}{4}
% 设置章、节、小节、小小节的间距
\titleformat{\chapter}[hang]{\normalfont\xiaosan\hei\sf}{\xiaosan\thechapter}{10pt}{\xiaosan}
\titlespacing{\chapter}{0pt}{-3ex  plus .1ex minus .2ex}{3.3ex}
\titleformat{\section}[hang]{\sihao\hei\sf}{\sihao\thesection}{0.5em}{}{}
\titlespacing{\section}{0pt}{0.5em}{0.5em}
\titleformat{\subsection}[hang]{\xiaosi\hei\sf}{\xiaosi\thesubsection}{0.5em}{}{}
\titlespacing{\subsection}{0pt}{0.5em}{0.3em}
\titleformat{\subsubsection}[hang]{\hei\sf}{\thesubsubsection}{0.5em}{}{}
\titlespacing{\subsubsection}{0pt}{0.3em}{0pt}
% 缩小目录中各级标题之间的缩进
\dottedcontents{chapter}[0.32cm]{\vspace{0.2em}}{1.0em}{5pt}
\dottedcontents{section}[1.32cm]{}{1.8em}{5pt}
\dottedcontents{subsection}[2.32cm]{}{2.7em}{5pt}
\dottedcontents{subsubsection}[3.32cm]{}{3.4em}{5pt}

% 段落之间的竖直距离
\setlength{\parskip}{1.2pt}
% 段落缩进
\setlength{\parindent}{24pt}
% 定义行距
\renewcommand{\baselinestretch}{1.27}
% 参考文献条目间行间距
\setlength{\bibsep}{2pt}

%%%%%%%%%%%%%%%%%%%%%%%%%%%%%%%%%%%%%%%%%%%%%%%%%%%%%%%%%%%%%%%%%%%%%% 
% 页眉页脚设置
%%%%%%%%%%%%%%%%%%%%%%%%%%%%%%%%%%%%%%%%%%%%%%%%%%%%%%%%%%%%%%%%%%%%%% 

\newcommand{\makeheadrule}{%
  \makebox[0pt][l]{\rule[.7\baselineskip]{\headwidth}{0.5pt}}%
  \vskip-.8\baselineskip}

\makeatletter
\renewcommand{\headrule}{%
  {\if@fancyplain\let\headrulewidth\plainheadrulewidth\fi
    \makeheadrule}}

\pagestyle{fancyplain}

\fancyhf{}
\fancyhead[CO]{\song\wuhao{大连理工大学硕士学位论文}}
\fancyhead[CE]{\song\wuhao\@ctitle}
\fancyfoot[C,C]{\xiaowu$-$~\thepage~$-$}

% Clear Header Style on the Last Empty Odd pages
\makeatletter
\def\cleardoublepage{\clearpage\if@twoside \ifodd\c@page\else%
  \hbox{}%
  \thispagestyle{empty}%              % Empty header styles
  \newpage%
  \if@twocolumn\hbox{}\newpage\fi\fi\fi}



%%%%%%%%%%%%%%%%%%%%%%%%%%%%%%%%%%%%%%%%%%%%%%%%%%%%%%%%%%%%%%%%%%%%%% 
% 列表环境设置
%%%%%%%%%%%%%%%%%%%%%%%%%%%%%%%%%%%%%%%%%%%%%%%%%%%%%%%%%%%%%%%%%%%%%% 
% \let\orig@Itemize =\itemize
% \let\orig@Enumerate =\enumerate
% \let\orig@Description =\description

% \def\Myspacing{\itemsep=1ex \topsep=-4ex \partopsep=-2ex \parskip=-1ex \parsep=2ex}
% \def\newitemsep{
% \renewenvironment{itemize}{\orig@Itemize\Myspacing}{\endlist}
% \renewenvironment{enumerate}{\orig@Enumerate\Myspacing}{\endlist}
% \renewenvironment{description}{\orig@Description\Myspacing}{\endlist}
% }
%   \def\olditemsep{
%   \renewenvironment{itemize}{\orig@Itemize}{\endlist}
%   \renewenvironment{enumerate}{\orig@Enumerate}{\endlist}
%   \renewenvironment{description}{\orig@Description}{\endlist}
% }
%   \renewcommand{\labelenumi}{(\arabic{enumi})}
%   \newitemsep

%%%%%%%%%%%%%%%%%%%%%%%%%%%%%%%%%%%%%%%%%%%%%%%%%%%%%%%%%%%%%%%%%%%%%%   
%   其他设置
%%%%%%%%%%%%%%%%%%%%%%%%%%%%%%%%%%%%%%%%%%%%%%%%%%%%%%%%%%%%%%%%%%%%%%   
%   增加 \ucite 命令使显示的引用为上标形式
%   \newcommand{\ucite}[1]{$^{\mbox{\scriptsize \cite{#1}}}$}

%%%%%%%%%%%%%%%%%%%%%%%%%%%%%%%%%%%%%%%%%%%%%%%%%%%%%%%%%%%%%%%%%%%%%%   
%   图形表格
%%%%%%%%%%%%%%%%%%%%%%%%%%%%%%%%%%%%%%%%%%%%%%%%%%%%%%%%%%%%%%%%%%%%%%   
\renewcommand{\figurename}{图}
\renewcommand{\tablename}{表}
% \captionstyle{\centering}
% \hangcaption
\captiondelim{\hspace{1em}}
\captionnamefont{\zhongwu}
\captiontitlefont{\zhongwu}
\setlength{\abovecaptionskip}{0pt}
\setlength{\belowcaptionskip}{0pt}


\newcommand{\tablepage}[2]{\begin{minipage}{#1}\vspace{0.5ex} #2 \vspace{0.5ex}\end{minipage}}
\newcommand{\returnpage}[2]{\begin{minipage}{#1}\vspace{0.5ex} #2 \vspace{-1.5ex}\end{minipage}}


%%%%%%%%%%%%%%%%%%%%%%%%%%%%%%%%%%%%%%%%%%%%%%%%%%%%%%%%%%%%%%%%%%%%%% 
% 定义题头格言的格式
%%%%%%%%%%%%%%%%%%%%%%%%%%%%%%%%%%%%%%%%%%%%%%%%%%%%%%%%%%%%%%%%%%%%%% 

\newsavebox{\AphorismAuthor}
\newenvironment{Aphorism}[1]
{\vspace{0.5cm}\begin{sloppypar} \slshape
    \sbox{\AphorismAuthor}{#1}
    \begin{quote}\small\itshape }
    {\\ \hspace*{\fill}------\hspace{0.2cm} \usebox{\AphorismAuthor}
    \end{quote}
  \end{sloppypar}\vspace{0.5cm}}

% 自定义一个空命令,用于注释掉文本中不需要的部分。
\newcommand{\comment}[1]{}

% This is the flag for longer version
\newcommand{\longer}[2]{#1}

\newcommand{\ds}{\displaystyle}

% define graph scale
\def\gs{1.0}

%%%%%%%%%%%%%%%%%%%%%%%%%%%%%%%%%%%%%%%%%%%%%%%%%%%%%%%%%%%%%%%%%%%%%%%%%%%%%%%% 
% 封面摘要
%%%%%%%%%%%%%%%%%%%%%%%%%%%%%%%%%%%%%%%%%%%%%%%%%%%%%%%%%%%%%%%%%%%%%%%%%%%%%%%% 
\def\cdegree#1{\def\@cdegree{#1}}\def\@cdegree{}
\def\ctitle#1{\def\@ctitle{#1}}\def\@ctitle{}
\def\caffil#1{\def\@caffil{#1}}\def\@caffil{}
\def\csubject#1{\def\@csubject{#1}}\def\@csubject{}
\def\cauthor#1{\def\@cauthor{#1}}\def\@cauthor{}
\def\cauthorno#1{\def\@cauthorno{#1}}\def\@cauthorno{}
\def\csupervisor#1{\def\@csupervisor{#1}}\def\@csupervisor{}
\def\cdate#1{\def\@cdate{#1}}\def\@cdate{}
\long\def\cabstract#1{\long\def\@cabstract{#1}}\long\def\@cabstract{}
\def\ckeywords#1{\def\@ckeywords{#1}}\def\@ckeywords{}
\def\etitle#1{\def\@etitle{#1}}\def\@etitle{}
\long\def\eabstract#1{\long\def\@eabstract{#1}}\long\def\@eabstract{}
\def\ekeywords#1{\def\@ekeywords{#1}}\def\@ekeywords{}

% 封面
\def\makecover{
  \begin{titlepage}
    \newpage
    \thispagestyle{empty}
    \begin{center}
      \parbox[t][4.40cm][c]{\textwidth}
      {
        \begin{center}
          {\xiaoyi\song\textbf{\@cdegree}\\}
          \vspace{1.34cm}
          {\erhao\hei\textsf{\@ctitle}\\}
          \vspace{0.13cm}
          {\sanhao\textbf{\@etitle}\\}
        \end{center}
      }
      \parbox[b][10.39cm][c]{\textwidth}
      {
        \vspace{8.05cm}
        \begin{center}
          {
            \xiaosan\song
            \begin{tabular}{p{0.6cm}p{6.4em}@{\extracolsep{0.5em}}lc}
              \vspace{-0.41cm}
              ~ & 作 \hfill 者 \hfill 姓 \hfill 名:& \fixedunderlineleft[6.08cm]{\@cauthor} & \\
              \vspace{-0.41cm}
              ~ & 学 \hfill 科 \hfill 、\hfill 专 \hfill 业:& \fixedunderlineleft[6.08cm]{\@csubject} & \\
              \vspace{-0.41cm}
              ~ & 学 \hfill 号:& \fixedunderlineleft[6.08cm]{\@cauthorno} & \\
              \vspace{-0.41cm}
              ~ & 指 \hfill 导 \hfill 教 \hfill 师:& \fixedunderlineleft[6.08cm]{\@csupervisor} & \\
              \vspace{-0.41cm}
              ~ & 完 \hfill 成 \hfill 日 \hfill 期:& \fixedunderlineleft[6.08cm]{\@cdate} & \\
              \vspace{-0.41cm}
            \end{tabular}
          }
        \end{center}
      }
      
      \vspace{2.34cm}
      
      {
     	\xiaoer\xkai
        大连理工大学\\
        \vspace{0.24cm}
        \xiaosi\song
        Dalian University of Technology
      }
    \end{center}
    \cleardoublepage
  \end{titlepage}
}

% 独创性说明
\def\originality{
  \renewcommand{\baselinestretch}{1.61}
  \newpage
  \thispagestyle{empty}
  \begin{center}
    \parbox[t][1.52cm][c]{\textwidth}
    {\xiaoer\song\centerline{大连理工大学学位论文独创性声明}}
    \parbox[t][7.68cm][c]{\textwidth}
    {
      \sihao\fs\noindent
        作者郑重声明:所呈交的学位论文,是本人在导师的指导下进行研究工作所取得的成果。%
      尽我所知,除文中已经注明引用内容和致谢的地方外,%
      本论文不包含其他个人或集体已经发表的研究成果,%
      也不包含其他已申请学位或其他用途使用过的成果。%
      与我一同工作的同志对本研究所做的贡献均已在论文中做了明确的说明并表示了谢意。%
      
      \vspace{0.47cm}\noindent
        若有不实之处,本人愿意承担相关法律责任。
    }
    \parbox[t][0.58cm][c]{\textwidth}
    {\sihao\fs 学{\hfill}位{\hfill}论{\hfill}文{\hfill}题{\hfill}目{\hfill}:\fixedunderlinecenter[12.6cm]{\@ctitle}}
    \parbox[t][1.22cm][c]{\textwidth}
    {\sihao\fs 作{\hfill}者{\hfill}签{\hfill}名{\hfill}:\fixeddistanceleft[12.6cm]{\underline{\hspace{5.8cm}}\hfill
	日期:\underline{\hspace{1.4cm}}~年~{\underline{\hspace{0.7cm}}~月~{\underline{\hspace{0.7cm}}~日~}}}}
  \end{center}
  \renewcommand{\baselinestretch}{1.27}
  \cleardoublepage
}

\def\authorization
{
  \begin{center}
    \sanhao\hei~大连理工大学学位论文版权使用授权书~
  \end{center}
  
  \sihao\fs\noindent
    本人完全了解学校有关学位论文知识产权的规定,%
  在校攻读学位期间论文工作的知识产权属于大连理工大学,允许论文被查阅和借阅。%
  学校有权保留论文并向国家有关部门或机构送交论文的复印件和电子版,%
  可以将本学位论文的全部或部分内容编入有关数据库进行检索,%
  可以采用影印、缩印、或扫描等复制手段保存和汇编本学位论文。%
  \\
  
  \vspace{0.2cm}\noindent
  学{\hfill}位{\hfill}论{\hfill}文{\hfill}题{\hfill}目{\hfill}:\fixedunderlinecenter[12.6cm]{\@ctitle}\\
  作{\hfill}者{\hfill}签{\hfill}名{\hfill}:\fixeddistanceleft[12.6cm]{\underline{\hspace{5.8cm}}\hfill
    日期:\underline{\hspace{1.4cm}}~年~{\underline{\hspace{0.7cm}}~月~{\underline{\hspace{0.7cm}}~日~}}}\\
  导{\hfill}师{\hfill}签{\hfill}名{\hfill}:\fixeddistanceleft[12.6cm]{\underline{\hspace{5.8cm}}\hfill
    日期:\underline{\hspace{1.4cm}}~年~{\underline{\hspace{0.7cm}}~月~{\underline{\hspace{0.7cm}}~日~}}}\\
  \renewcommand{\baselinestretch}{1.27}
}

\def\makeabstract{
  \defaultfont
  \chapter*{\hfill 摘  要 \hfill}
  \addcontentsline{toc}{chapter}{摘  要}
  \setcounter{page}{1}
  \@cabstract
  \vspace{0.53cm}
  
  \noindent {\hei{关键词:{\fs\@ckeywords}}}
  
  \defaultfont
  \cleardoublepage
  \chapter*{}
  \addcontentsline{toc}{chapter}{Abstract}
  \vspace{-1.40cm}
  \begin{center}
    {\xiaosan\textrm{\@etitle}}
  \end{center}
  \vspace{-0.35cm}
  \begin{center}
    {
      \xiaosan{Abstract}\\
    }
  \end{center}
  \vspace{0.12cm}
  
  \@eabstract
  
  \vspace{0.55cm}
  
  \noindent {\textbf{Key Words:}}~~{\textsf{\@ekeywords}}
  \cleardoublepage
}

\makeatletter
\def\hlinewd#1{%
  \noalign{\ifnum0=`}\fi\hrule \@height #1 \futurelet
  \reserved@a\@xhline}
\makeatother

% 定义索引生成
\def\generateindex
{
  \addcontentsline{toc}{chapter}{\indexname}
  \printindex
  \cleardoublepage
}

\raggedbottom 