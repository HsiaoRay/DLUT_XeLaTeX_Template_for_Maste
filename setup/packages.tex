% !TEX TS-program = XeLaTeX
% !TEX encoding = UTF-8 Unicode

%%%%%%%%%%%%%%%%%%%%%%%%%%%%%%%%%%%%%%%%%%%%%%%%%%%%%%%%%%%%%%%%%%%% 
% 
% 大连理工大学硕士论文 XeLaTeX 模版 —— 宏包配置文件 packages.tex
% 版本:0.8
% 最后更新:2012.04.04
% 修改者:Yuri (E-mail: yuri_1985@163.com)
% 修订者:whufanwei(dutfanwei@qq.com) 
% 编译环境1:Ubuntu 12.04 + TeXLive 2011 + Emacs
% 编译环境2:Windows 7 + CTeX v2.9.2.164 + WinEdit
%%%%%%%%%%%%%%%%%%%%%%%%%%%%%%%%%%%%%%%%%%%%%%%%%%%%%%%%%%%%%%%%%%%%% 

% 页面设置
\usepackage[body={16.1cm, 22.2cm}]{geometry}
\usepackage{indentfirst}                         % 首行缩进宏包
\usepackage[sf]{titlesec}                        % 控制标题的宏包
\usepackage{titletoc}                            % 控制目录的宏包
\usepackage{fancyhdr}                            % 自定义页眉页脚
\usepackage{fancyref}                            % 引用链接属性
\usepackage[perpage,symbol]{footmisc}            % 脚注控制
\usepackage{layouts}                             % 打印当前页面格式的宏包
\usepackage{paralist}                            % 一种换行不缩进的列表格式,asparaenum,inparaenum 等
\usepackage[shortlabels]{enumitem}               % 列表格式
\setlist[enumerate]{(1),itemsep=-5pt,topsep=0mm,labelindent=\parindent,leftmargin=*}
\usepackage{fancyvrb}                            % 原样输出
\usepackage[amsmath,thmmarks,hyperref]{ntheorem} % 定理类环境宏包
\usepackage{type1cm}                             % 控制字体的大小

% 表格处理
\usepackage{booktabs}   % 三线表
\usepackage{multirow}   % 表格多行处理
\usepackage{diagbox}    % 斜线表头
\usepackage{tabularx}   % 表格折行
\usepackage{siunitx}    % 国际单位,小数点对齐
\sisetup{inter-unit-product = { }\cdot{ }} 

% 参考文献

% 图形相关
\usepackage{graphicx}         % 请在引用图片时务必给出后缀名
\usepackage[x11names]{xcolor} % 支持彩色
\usepackage[below]{placeins}  % 浮动图形控制宏包
\usepackage{rotating}	      % 图形和表格的控制
\usepackage{ccaption}         % 插图表格的双语标题
\usepackage{setspace}         % 定制表格和图形的多行标题行距
\usepackage{subcaption}       %子图


\usepackage{listings}         % 源代码展示
\lstset{%
  language=TeX,
  defaultdialect=empty,
  basicstyle=\ttfamily\small,
  backgroundcolor=\color{LightSteelBlue1},
  keywordstyle=\color{blue},
  showspaces=false,
  showstringspaces=false,
  showtabs=false,
  tabsize=2,breakatwhitespace=false,
  columns=flexible}

% 其他
\usepackage{calc}   % 在 tex 文件中具有一些计算功能,主要用在页面控制。
\usepackage[xetex,
bookmarksnumbered=true,
bookmarksopen=true,
colorlinks=true,
% pdfborder={0 0 1},
citecolor=blue,
linkcolor=blue,
anchorcolor=green,
urlcolor=magenta,
breaklinks=true,
CJKbookmarks=true,
]{hyperref}


\usepackage[numbers,sort&compress,square,super]{natbib} %参考文献
\usepackage{hypernat}
\usepackage{bibentry}
\addtolength{\bibsep}{-0.5em}              % 缩小参考文献间的垂直间距
\setlength{\bibhang}{2em}
\bibpunct{[}{]}{,}{s}{}{}











